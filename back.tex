% !TEX TS-program = XeLaTeX      Must use XeLaTeX to compile
% !TEX encoding = UTF-8 Unicode  File must be encoded at UTF-8

\documentclass{orly}

\begin{document}


%%% FRAME %%%%%%%%%%%%%%%%%%%%%%%%%%%%%%%%%%%%%%%%%%%%%%%%%%%%%%%%%%%%%%%%%%%%%

% perimeter
\begin{tikzpicture}[remember picture, overlay, shift={(current page.north west)}]
%    \draw[draw=black] (\xshift, -\yshift) rectangle ++(\pagewidth pt,-\pageheight pt);  % front cover
%    \draw[draw=black] (\xshift, -\yshift) rectangle ++(\coverwidth pt, -\coverheight pt);  % back cover
%    \draw[draw=black] (\xshift + 0.5*\coverwidth pt, -\yshift) rectangle ++(0, -\pageheight pt);  % back cover, midpoint
\end{tikzpicture}


\cutlines


%%% BACKCOVER %%%%%%%%%%%%%%%%%%%%%%%%%%%%%%%%%%%%%%%%%%%%%%%%%%%%%%%%%%%%%%%%%
% top banner
% centering text
% https://www.reddit.com/r/LaTeX/comments/s1sx1w/tikz_centering_text_from_a_node/
\begin{tikzpicture}[remember picture, overlay, shift={(current page.north west)}, xshift=\xshift, yshift=-\yshift]

    \node[anchor=north west] at ($(current page.north west) + (\xshift, -\yshift) + (\bmargin, -1*\bheight)$)
        {\backsubtitle};

    \node[anchor=north west] at ($(current page.north west) + (\xshift, -\yshift) + (0.4*\coverwidth pt, -1.8*\bheight)$)
        {\huge \oreilly O'RLY\small$^\textsuperscript{\textregistered}$};

    \filldraw [fill=\bcolor, draw=\bcolor] (\bmargin, -4*\bheight) rectangle ++(\bwidth, -\bheight);

    \node[anchor=north west] at ($(current page.north west) + (\xshift, -\yshift) + (\bmargin, -5.5*\bheight)$)
        {\huge \textbf Navigating Legacy Code};

    \node[anchor=north west, text width=\bwidth] at ($(current page.north west) + (\xshift, -\yshift) + (\bmargin, -7.75*\bheight)$)
        {\lipsum[3-4]};
\end{tikzpicture}

% barcode
% NOTE something is wrong with the coordinates of the barcode, it doesn't respect the units or location
% placing it through trial and error
\begin{pspicture}(0,0)
    \psbarcode[transy=-18.4cm, transx=0.4cm]{0080085}{includetext height=0.4 width=0.8}{upce}
\end{pspicture}



\end{document}
