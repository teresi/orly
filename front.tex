% !TEX TS-program = XeLaTeX      Must use XeLaTeX to compile
% !TEX encoding = UTF-8 Unicode  File must be encoded at UTF-8

% TODO use pst-barcode to add a fake barcode?
% https://mirror.math.princeton.edu/pub/CTAN/graphics/pstricks/contrib/pst-barcode/doc/pst-barcode-doc.pdf

\documentclass{article}
\usepackage{graphicx}
\usepackage{fontspec}
\usepackage{tikz}
\usepackage{pst-barcode}
\usetikzlibrary{calc}
\usepackage{lipsum}


\usepackage[rgb]{xcolor}

%\setmainfont{DejaVu Serif}
\setmainfont{FreeSerif}
\definecolor{linux}{HTML}{920F02}


\usepackage{xfp}

\usepackage[%
	papersize={8.5in,11in},%
]{geometry}

\pagenumbering{gobble}



\begin{document}

% TODO just convert everything to pt and then appt pt to the call in tikz

% Midori A5 notebook is 14.81 x 21.01 x 1.054 cm
\newcommand{\pttocm}{0.0352778}             % points per centimeter
\newcommand{\coverwidth}{14.8 / \pttocm}    % bookcover width
\newcommand{\coverheight}{21.0 / \pttocm}   % bookcover height
\newcommand{\spinewidth}{1.054 / \pttocm}   % bookcover spine width

\newcommand{\pagewidth}{\fpeval{\coverwidth}}  % combined cover width
\newcommand{\pageheight}{\coverheight}                           % combined cover height

%\newfontfamily\oreilly{VL PGothic}
\newfontfamily\oreilly{IPAexGothic}

% NB in centimeters
\newcommand{\bcolor}{linux}  % banner color
\newcommand{\bmargin}{1 cm}           % banner xoffset  % TODO in cm, convert to pt
\newcommand{\bheight}{0.4 cm}         % banner height   % TODO in cm, convert to pt
\newcommand{\bwidth}{\fpeval{(\coverwidth * 0.85 * \pttocm)}cm}

\newcommand{\xshift}{\fpeval{(\paperwidth - \pagewidth) * 0.5 * \pttocm}cm}
\newcommand{\yshift}{\fpeval{(\paperheight - \pageheight) * 0.5 * \pttocm}cm}

%%% FRAME %%%%%%%%%%%%%%%%%%%%%%%%%%%%%%%%%%%%%%%%%%%%%%%%%%%%%%%%%%%%%%%%%%%%%

% perimeter
\begin{tikzpicture}[remember picture, overlay, shift={(current page.north west)}]
%    \draw[draw=black] (\xshift, -\yshift) rectangle ++(\pagewidth pt,-\pageheight pt);  % front cover
%    \draw[draw=black] (\xshift, -\yshift) rectangle ++(\coverwidth pt, -\coverheight pt);  % back cover
%    \draw[draw=black] (\xshift + 0.5*\coverwidth pt, -\yshift) rectangle ++(0, -\pageheight pt);  % back cover, midpoint
\end{tikzpicture}


% cut lines, vertical
\newcommand{\cutoff}{24pt}
\begin{tikzpicture}[remember picture, overlay, shift={(current page.north west)}, xshift=\xshift]
    \draw[draw=black] (0, -\yshift + \cutoff)--(0, 0);
    \draw[draw=black] (\pagewidth pt, -\yshift + \cutoff)--(\pagewidth pt, 0);

    \draw[draw=black] (0, -\yshift-\coverheight pt -\cutoff)  --(0, -\paperheight);
    \draw[draw=black] (\pagewidth pt, -\yshift-\coverheight pt -\cutoff)  --(\pagewidth pt, -\paperheight);
\end{tikzpicture}

% cut lines, horizontal
\begin{tikzpicture}[remember picture, overlay, shift={(current page.north west)}, yshift=-\yshift]
    \draw[draw=black] (\xshift -\cutoff, 0)--(0, 0);
    \draw[draw=black] (\xshift + \pagewidth + \cutoff, 0)--(\paperwidth, 0);

    \draw[draw=black] (\xshift -\cutoff, -\pageheight pt)--(0, -\pageheight pt);
    \draw[draw=black] (\xshift + \pagewidth + \cutoff, -\pageheight pt)--(\paperwidth, -\pageheight pt);
\end{tikzpicture}


%%% FRONT %%%%%%%%%%%%%%%%%%%%%%%%%%%%%%%%%%%%%%%%%%%%%%%%%%%%%%%%%%%%%%%%%%%%%

\newcommand{\xfrontshift}{\fpeval{\xshift}pt}

\begin{tikzpicture}[remember picture, overlay, shift={(current page.north west)}, xshift=\xfrontshift, yshift=-\yshift]
    \filldraw [fill=\bcolor, draw=\bcolor] (\bmargin, 0) rectangle ++(\bwidth, -\bheight);
\end{tikzpicture}

% animal
\begin{tikzpicture}[remember picture, overlay, shift={(current page.north west)} ]
    \node[anchor=north] at ($(current page.north west) + (\xfrontshift, -\yshift) + (0.5*\coverwidth pt, -4.0*\bheight)$)
    {\includegraphics[height=9cm]{./img/sunwave}};
\end{tikzpicture}

% title
\newcommand{\toffset}{\fpeval{\coverheight * 0.55 * -1}}
\newcommand{\theight}{\fpeval{round(\coverheight * 0.3 )}}
\begin{tikzpicture}[remember picture, overlay, shift={(current page.north west)}, xshift=\xfrontshift, yshift=-\yshift]

    \filldraw [fill=\bcolor, draw=white] (\bmargin, \toffset pt) rectangle ++(\bwidth, -\theight pt);

    \node[anchor=north west, font=\itshape, scale=1.1] at ($(current page.north west) + (\xfrontshift, -\yshift) + (\bmargin*1.1, \toffset+1.1*\bmargin)$)
        {\Huge Navigating};

    \node[anchor=north west, text=white, scale=3, text width=0.9*\bwidth] at ($(current page.north west) + (\xfrontshift, -\yshift) + (\bmargin*1.2, \toffset pt - 0.6*\bmargin)$)
        {\Huge Legacy\\ Code};

    \node[anchor=north west, font=\itshape] at ($(current page.north west) + (\xfrontshift, -\yshift) + (\bwidth*0.77, \toffset-\theight-0.65*\bheight)$)
        {\Large advanced rm -rf *};

    \node[anchor=north west] at ($(current page.south west) + (\xfrontshift, \yshift) + (\bmargin, 3*\bheight)$)
        {\huge \oreilly O'RLY\small$^\textsuperscript{\textregistered}$};

    % TODO there must be a better way; how to justify text in the node?
    \node[anchor=north west, font=\itshape] at ($(current page.south west) + (\xfrontshift, \yshift) + (\bwidth*0.72, 3*\bheight)$)
        {\huge Michael Teresi};

    \node[anchor=north west, font=\itshape] at ($(current page.north west) + (\xfrontshift, -\yshift) + (\bwidth*0.3, -\bheight)$)
        {\large What has never worked can never be lost};

\end{tikzpicture}


\end{document}
