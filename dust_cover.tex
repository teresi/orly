% !TEX TS-program = XeLaTeX      Must use XeLaTeX to compile
% !TEX encoding = UTF-8 Unicode  File must be encoded at UTF-8

% TODO use pst-barcode to add a fake barcode?
% https://mirror.math.princeton.edu/pub/CTAN/graphics/pstricks/contrib/pst-barcode/doc/pst-barcode-doc.pdf

\documentclass[
    papersize={17in,11in}
]{orly}

\renewcommand{\pagewidth}{\fpeval{\coverwidth * 2 + \spinewidth}}  % combined cover width
\renewcommand{\pageheight}{\coverheight}                            % combined cover height


\begin{document}

%%% FRAME %%%%%%%%%%%%%%%%%%%%%%%%%%%%%%%%%%%%%%%%%%%%%%%%%%%%%%%%%%%%%%%%%%%%%

% perimeter
\begin{tikzpicture}[remember picture, overlay, shift={(current page.north west)}]
%    \draw[draw=black] (\xshift, -\yshift) rectangle ++(\pagewidth pt,-\pageheight pt);  % front cover
%    \draw[draw=black] (\xshift, -\yshift) rectangle ++(\coverwidth pt, -\coverheight pt);  % back cover
%    \draw[draw=black] (\xshift + 0.5*\coverwidth pt, -\yshift) rectangle ++(0, -\pageheight pt);  % back cover, midpoint
\end{tikzpicture}


% cut lines, vertical
\begin{tikzpicture}[remember picture, overlay, shift={(current page.north west)}, xshift=\xshift]
    \draw[draw=black] (0, -\yshift + 16pt)--(0, 0);
    \draw[draw=black] (\pagewidth pt, -\yshift + 16pt)--(\pagewidth pt, 0);

    \draw[draw=black] (0, -\yshift-\coverheight pt -16pt)  --(0, -\paperheight);
    \draw[draw=black] (\pagewidth pt, -\yshift-\coverheight pt -16pt)  --(\pagewidth pt, -\paperheight);
\end{tikzpicture}

% cut lines, horizontal
\begin{tikzpicture}[remember picture, overlay, shift={(current page.north west)}, yshift=-\yshift]
    \draw[draw=black] (\xshift -16pt, 0)--(0, 0);
    \draw[draw=black] (\xshift + \pagewidth + 16pt, 0)--(\paperwidth, 0);

    \draw[draw=black] (\xshift -16pt, -\pageheight pt)--(0, -\pageheight pt);
    \draw[draw=black] (\xshift + \pagewidth + 16pt, -\pageheight pt)--(\paperwidth, -\pageheight pt);
\end{tikzpicture}


%%% BACKCOVER %%%%%%%%%%%%%%%%%%%%%%%%%%%%%%%%%%%%%%%%%%%%%%%%%%%%%%%%%%%%%%%%%
% top banner
% centering text
% https://www.reddit.com/r/LaTeX/comments/s1sx1w/tikz_centering_text_from_a_node/
\begin{tikzpicture}[remember picture, overlay, shift={(current page.north west)}, xshift=\xshift, yshift=-\yshift]

    \node[anchor=north west] at ($(current page.north west) + (\xshift, -\yshift) + (\bmargin, -1*\bheight)$)
        {Unix Programming};

    \node[anchor=north west] at ($(current page.north west) + (\xshift, -\yshift) + (0.4*\coverwidth pt, -1.8*\bheight)$)
        {\huge \oreilly O'RLY\small$^\textsuperscript{\textregistered}$};

    \filldraw [fill=\bcolor, draw=\bcolor] (\bmargin, -4*\bheight) rectangle ++(\bwidth, -\bheight);

    \node[anchor=north west] at ($(current page.north west) + (\xshift, -\yshift) + (\bmargin, -5.5*\bheight)$)
        {\huge \textbf Navigating Legacy Code};

    \node[anchor=north west, text width=\bwidth] at ($(current page.north west) + (\xshift, -\yshift) + (\bmargin, -7.75*\bheight)$)
        {\lipsum[3-4]};
\end{tikzpicture}

% barcode
% NOTE something is wrong with the coordinates of the barcode, it doesn't respect the units or location
% placing it through trial and error
\begin{pspicture}(0,0)
    \psbarcode[transy=-18.4cm, transx=0.4cm]{0080085}{includetext height=0.4 width=0.8}{upce}
\end{pspicture}


%%% SPINE %%%%%%%%%%%%%%%%%%%%%%%%%%%%%%%%%%%%%%%%%%%%%%%%%%%%%%%%%%%%%%%%%%%%%
\newcommand{\xspineshift}{\fpeval{\xshift + \coverwidth}pt}

\begin{tikzpicture}[remember picture, overlay, shift={(current page.north west)}, xshift=\xspineshift, yshift=-\yshift]
    % title banner
    \filldraw [fill=\bcolor, draw=\bcolor] (0, -1.5cm-24pt) rectangle ++(\spinewidth pt, -\coverheight pt + 2.5cm);
    \node[anchor=north west, text=white, rotate=-90] at ($(current page.north west) + (\xspineshift + 0.9*\spinewidth, -2.5*\yshift) + (0, 0)$)
        {\huge Navigating Legacy Code \quad \quad \large Michael Teresi};

    % animal
    % NOTE this image does not line up with the rectangles, is there a hidden space? Had to hard code this
    \node[anchor=north] at ($(current page.north west) + (\xspineshift+0.5*\spinewidth, -\yshift-8.8pt)$)
        {\includegraphics[height=1.5cm,width=1.1 cm, trim={0 90 0 0}, clip]{./img/sunwave}};
    \filldraw [fill=black, draw=black] (0, 0) rectangle ++(\spinewidth pt, -12 pt);
    \filldraw [fill=black, draw=black] (0, -1.5cm -12pt) rectangle ++(\spinewidth pt, -12 pt);


    \node[anchor=north, text=white] at ($(current page.north west) + (\xspineshift+0.5*\spinewidth, -1.5*\yshift-3.5 pt)$)
        {\normalsize \textbf 1st Ed.};

    % orly banner
    \filldraw [fill=black, draw=black] (0, -\coverheight pt + 2.5cm) rectangle ++(\spinewidth pt, -2.5cm);
    \node[anchor=north west, text=white, rotate=-90] at ($(current page.north west) + (\xspineshift + 0.83*\spinewidth, -6.38*\yshift)$)
        {\large \oreilly O'RLY\small$^\textsuperscript{\textregistered}$};

\end{tikzpicture}



%%% FRONTCOVER %%%%%%%%%%%%%%%%%%%%%%%%%%%%%%%%%%%%%%%%%%%%%%%%%%%%%%%%%%%%%%%%

\newcommand{\xfrontshift}{\fpeval{\xshift + \coverwidth + \spinewidth}pt}

\begin{tikzpicture}[remember picture, overlay, shift={(current page.north west)}, xshift=\xfrontshift, yshift=-\yshift]
    \filldraw [fill=\bcolor, draw=\bcolor] (\bmargin, 0) rectangle ++(\bwidth, -\bheight);
    %\filldraw [fill=\bcolor, draw=white] (\bmargin, -\coverheight pt) rectangle ++(\bwidth, \bheight);
\end{tikzpicture}

%\begin{tikzpicture}[remember picture, overlay, shift={(current page.north west)}, xshift=\xshift+\pagewidth, yshift=-\yshift]
%    \filldraw [fill=black, draw=black, rotate=-45] (-2cm, -2cm) rectangle ++(4cm, 1cm);
%\end{tikzpicture}


% animal
\begin{tikzpicture}[remember picture, overlay, shift={(current page.north west)} ]
    \node[anchor=north] at ($(current page.north west) + (\xfrontshift, -\yshift) + (0.25*\pagewidth pt, -4.0*\bheight)$)
    {\includegraphics[height=9cm]{./img/sunwave}};
\end{tikzpicture}

% title
\newcommand{\toffset}{\fpeval{\coverheight * 0.55 * -1}}
\newcommand{\theight}{\fpeval{round(\coverheight * 0.3 )}}
\begin{tikzpicture}[remember picture, overlay, shift={(current page.north west)}, xshift=\xfrontshift, yshift=-\yshift]

    \filldraw [fill=\bcolor, draw=white] (\bmargin, \toffset pt) rectangle ++(\bwidth, -\theight pt);

    \node[anchor=north west, font=\itshape, scale=1.1] at ($(current page.north west) + (\xfrontshift, -\yshift) + (\bmargin*1.1, \toffset+1.1*\bmargin)$)
        {\Huge Navigating};

    \node[anchor=north west, text=white, scale=3, text width=0.9*\bwidth] at ($(current page.north west) + (\xfrontshift, -\yshift) + (\bmargin*1.2, \toffset pt - 0.6*\bmargin)$)
        {\Huge Legacy\\ Code};

    \node[anchor=north west, font=\itshape] at ($(current page.north west) + (\xfrontshift, -\yshift) + (\bwidth*0.77, \toffset-\theight-0.65*\bheight)$)
        {\Large advanced rm -rf *};

    \node[anchor=north west] at ($(current page.south west) + (\xfrontshift, \yshift) + (\bmargin, 3*\bheight)$)
        {\huge \oreilly O'RLY\small$^\textsuperscript{\textregistered}$};

    % TODO there must be a better way; how to justify text in the node?
    \node[anchor=north west, font=\itshape] at ($(current page.south west) + (\xfrontshift, \yshift) + (\bwidth*0.72, 3*\bheight)$)
        {\huge Michael Teresi};

    \node[anchor=north west, font=\itshape] at ($(current page.north west) + (\xfrontshift, -\yshift) + (\bwidth*0.3, -\bheight)$)
        {\large What has never worked can never be lost};

\end{tikzpicture}


% TODO add O'REILLY
% TODO add author


%{\fontfamily{ppl}\selectfont ... }


\end{document}
